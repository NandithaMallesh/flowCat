\documentclass[11pt,a4paper]{article}
\usepackage[margin=2cm]{geometry}

%% adding graphics
\usepackage{graphicx}
$$%% nicer tables
\usepackage{booktabs}
%% rotate pages for landscape
\usepackage{pdflscape}
%% nice grid layouts for latex figures
\usepackage{subcaption}

\begin{document}

\title{Automatic Flow-based Classification of B-Cell Lymphoma}

\section{Introduction}

Flowcytometry aids in the detection and classification of lymphatic disorders, but current multi-channel flow data is hampered by the limitations of human interpretability.
Statistical preprocessing promises to enable the description of higher-dimensional relationships in measured events.
Extending previous work on preprocessing and classification of flowcytometry data, we created a proof-of-concept approach for unattended debris-removal and classification on a large dataset of B-Cell-Lymphoma.

\section{Aims}



\section{Methods}

A self-organizing map is trained on a sample of cases from all cohorts, creating a reference map of the data, we refer to as \emph{consensus-SOM}.
Invididual samples are afterwards sampled into the trained SOM and the distribution utilized for the classification into diagnoses in a sequential neural network.

In order to focus the map on a cell population in question individual samples were preprocessed by automatically selecting a CD45 positive and side-scatter negative population using a density based clustering of individually generated SOMs.

%% som based algorithm
%% consensus SOM
%% debris gating based on dbscan

\section{Results}

%% tsne results and som for normal vs cll
%% pregating w/ dbscan
%% neural net classification normal cll
%% results for xxx different lymphoma entities
%% clustering joint entities, similarity in classification

\section{Conclusions}

\end{document}
