\documentclass[11pt,a4paper]{article}
\usepackage[margin=2cm]{geometry}

%% adding graphics
\usepackage{graphicx}
%% nicer tables
\usepackage{booktabs}
%% rotate pages for landscape
\usepackage{pdflscape}
%% nice grid layouts for latex figures
\usepackage{subcaption}

\begin{document}

\title{Automatic Flow-based Classification of B-Cell Lymphoma}


\begin{abstract}
   Advances in the computational analysis of flow cytometry data promise better understanding of high-dimensional datasets and direct prediction of external variables.
   We developed an approach for automated classification of B-cell lymphoma based on interpretation of self-organizing maps using convolutional neural networks.
\end{abstract}


\section{Background}

%% advances in flow computation nature 2013
%% from one side better understanding of the possibility of detailed statistic
%% modeling of populations, on the other side apporoaches similar to flowsom
%% which seem to be more topology preserving


%% main questions for us:

%% main train of though: classification of lymphoma is complex -> machine
%% learning -> doability, -> weightings in the network reflect real differences

%% though outline: why does flowSOM work better than approaches that
%% seem to have a better statsitical grasp on the issue of flow cytometry?
%% 1. human analysis of flow cytometry data has not been statistical
%% 2. primary 2d plot interpretation is mainly visual, primarily based on
%% intuition of the graphical display of density and easily manipulated by that
%% --> complex hypotheses might be needed to capture the classification criteria
%%       which in turn leads itself to machine learning / neural networks
%% 3. topology preservation might be most important (back it up!!!)


\end{document}
